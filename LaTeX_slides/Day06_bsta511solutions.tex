%\documentclass[12pt,letterpaper]{article}
\documentclass[12pt]{amsart}

\usepackage[left=1in, right=1in, top=.5in, bottom=.5in]{geometry}
\usepackage{latexsym,amssymb,amsmath,amsthm,amsopn,verbatim, mathpazo, graphicx, lmodern, hyperref}
%\usepackage{latexsym,amssymb,amsmath,amsthm,amsopn,verbatim, mathpazo, graphicx}
\newcommand{\vs}{\vskip.5cm}
\newcommand{\hs}{\hskip1cm}
\newcommand{\ds}{\displaystyle}
\setlength{\parindent}{0pt}


\usepackage[T1]{fontenc}
\usepackage{libertine}
\renewcommand*\familydefault{\sfdefault}  %% Only if the base font of the document is to be sans serif


\newtheorem{theorem}{Theorem}[section]
\newtheorem{corollary}{Corollary}[theorem]
\newtheorem{lemma}[theorem]{Lemma}
\newtheorem{definition}[theorem]{Definition}
\newtheorem{example}[theorem]{Example}

\newcommand\indep{\protect\mathpalette{\protect\independenT}{\perp}}
\def\independenT#1#2{\mathrel{\rlap{$#1#2$}\mkern2mu{#1#2}}}

\begin{document}


%----------------------------------------------------------------------------
\setcounter{section}{2}
\setcounter{subsection}{4}
\setcounter{theorem}{6}
Day 6 BSTA 511/611
{\huge  
\section*{Chapter 2: Probability (Part 2)}
}

%----------------------------------------------------------------------------


%----------------------------------------------------------------------------
%{\large 
%----------------------------------------------------------------------------


%----------------------------------------------------------------------------
%\begin{example}  \textbf{text.} \newline
%
%\begin{enumerate}
%\item 
%
%\textbf{Solution:} \newline
%\vspace{2cm}
%
%\item 
%
%\textbf{Solution:} \newline
%
%\vspace{2cm}
%
%\item 
%
%\textbf{Solution:} \newline
%
%
%\vspace{2cm}
%
%
%\item 
%
%\textbf{Solution:} \newline
%
%\end{enumerate}
%\end{example} 
%\newpage


%\newpage
%----------------------------------------------------------------------------

\begin{example}  {\large \textbf{How accurate is rapid testing for COVID-19?} }

{\large From the iHealth® website  \newline}
{\tiny\url{https://ihealthlabs.com/pages/ihealth-covid-19-antigen-rapid-test-details}}: \newline
{\large "Based on the results of a clinical study where the iHealth® COVID-19 Antigen Rapid Test was compared to an FDA authorized molecular SARS-CoV-2 test, iHealth® COVID-19 Antigen Rapid Test correctly identified 94.3\% of positive specimens and 98.1\% of negative specimens."  \newline

%Questions:\newline
Suppose you take the iHealth® rapid test. 
\begin{enumerate}
	\item What is the probability of a positive test result?
	\item What is the probability of having COVID-19 if you get a positive test result?
	\item What is the probability of not having COVID-19 if you get a negative test result?

\end{enumerate}
\vspace{0.5cm}
\textbf{What information were we given?}%\newline
%\vspace{0.5cm}

First, let's define our events of interest: 
\begin{itemize}
	\item $D$ = event one has disease (COVID-19)
	\item $D^c$ = event one does not have disease
	\item $T^+$ = event one tests positive for disease
	\item $T^-$ = event one tests negative for disease
\end{itemize}
\vspace{0.5cm}
Translate given information into mathematical notation:
\begin{itemize}
	\item Test correctly gives a positive result 94.3\% of the time:\newline
%	hide
	Sensitivity  = $\mathbb{P}(T^+|D) = 0.943$
	\newline (aka PPA = positive percent agreement)
\vspace{0.5cm}
	\item Test correctly gives a negative result 98.1\% of the time: \newline 
%	hide
	Specificity = $\mathbb{P}(T^-|D^c) = 0.981$ 
	\newline (aka NPA = negative percent agreement)
\vspace{0.5cm}
	\item What are all the possible scenarios of test results?
%	hide
	Make 2-way table of test results (rows) vs. disease status (columns)

\begin{tabular}{| c| l| c | c | c |}
  \hline
             & $D$               & $D^c$           & Total\\                         
  \hline
  $T^+$ & sensitivity      & false positive & $\neq 1$ \\
  $T^-$ & false negative & specificity & $\neq 1$ \\
    \hline
   Total  & 1              & 1           & \\                         
  \hline  
\end{tabular}


\end{itemize}

\newpage
Given: $\mathbb{P}(T^+|D) = 0.943$, $\ \ \mathbb{P}(T^-|D^c) = 0.981$ %\newline

\vspace{0.5cm}
\textbf{Solutions to questions} %\newline
\begin{enumerate}
\item What is the probability of a positive test result?

%\textbf{Solution:} \newline
\vspace{2cm}
% hide
\begin{itemize}
	\item Sample space with $D$ vs. $D^c$ and blob for $T^c$
	\item Law of Total Probability
	\item Need to know prevalence, $\mathbb{P}(D)$ \newline
	Assume $\mathbb{P}(D) = 0.000838\ \  (0.0838\%)$
	\item Tree Diagram for calculating total probability
	\item Answer: 0.01977431
\end{itemize}

\newpage
Given: $\mathbb{P}(T^+|D) = 0.943$, $\ \ \mathbb{P}(T^-|D^c) = 0.981$ %\newline

\vspace{0.5cm}
\item What is the probability of having COVID-19 if you get a positive test result?

%\textbf{Solution:} \newline


\vspace{2cm}
% hide
\begin{itemize}
	\item $PPV = \mathbb{P}(D | T^+)$
	\item conditional probability
	\item Assume $\mathbb{P}(D) = 0.000838\ \  (0.0838\%)$ (Multnomah Co on 10/12/2022)
	\item Bayes' rule
	\item Answer: 0.03996265
	$$\frac{0.943*0.000838}{0.943*0.000838 + (1-0.981)*(1-0.000838)}= 0.03996265$$
	\item If $\mathbb{P}(D) = 0.01\ \  (1\%)$, then answer = 0.3339235
	\item If $\mathbb{P}(D) = 0.1\ \  (10\%)$, then answer = 0.8464991
\end{itemize}

\newpage
Given: $\mathbb{P}(T^+|D) = 0.943$, $\ \ \mathbb{P}(T^-|D^c) = 0.981$ %\newline

\vspace{0.5cm}
\item What is the probability of not having COVID-19 if you get a negative test result?

%\textbf{Solution:} \newline


\vspace{2cm}
% hide
\begin{itemize}
	\item $NPV = \mathbb{P}(D^c | T^-)$
	\item conditional probability
	\item Assume $\mathbb{P}(D) = 0.000838\ \  (0.0838\%)$
	\item Bayes' rule
	\item Answer: 0.9999513
%	$$\frac{0.943*0.000838}{0.943*0.000838 + (1-0.981)*(1-0.000838)}= 0.9999513$$
	$$\frac{0.981*0.999162}{0.981*0.999162 + (1-0.943)*0.000838}= 0.9999513$$
	\item If $\mathbb{P}(D) = 0.01\ \  (1\%)$, then answer = 0.9994134
	\item If $\mathbb{P}(D) = 0.1\ \  (10\%)$, then answer = 0.9935854
\end{itemize}

\end{enumerate}
}
\end{example} 
{\large

\newpage
%----------------------------------------------------------------------------
\textbf{Bayes' Theorem (Section 2.2.5)}

\vspace{0.5cm}
In the previous examples we derived the formula for Bayes' Theorem.
\vspace{0.5cm}


\begin{theorem}[Bayes' Theorem]  

If the sample space $S$ can be split into disjoint events $A_1, A_2, ..., A_k$ that make up all possible outcomes in $S$, and if  $\mathbb{P}(A_i)>0$ for $i=1,\ldots, k$ and $\mathbb{P}(B)>0$, then


$$
\mathbb{P}(A_1 | B) = 
\frac{\mathbb{P}(B|A_1) \cdot \mathbb{P}(A_1)}
{\mathbb{P}(B|A_1) \cdot \mathbb{P}(A_1) + \mathbb{P}(B|A_2) \cdot \mathbb{P}(A_2) + ... + 
\mathbb{P}(B|A_k) \cdot \mathbb{P}(A_k)}
$$

\vspace{0.5cm}

\end{theorem}

\vspace{0.8cm}

Special case of Bayes' Theorem for sample space being split into $A$ and $A^c$: 

$$
\mathbb{P}(A | B) = 
\frac{\mathbb{P}(B|A) \cdot \mathbb{P}(A)}
{\mathbb{P}(B|A) \cdot \mathbb{P}(A) + \mathbb{P}(B|A^c) \cdot \mathbb{P}(A^c) }
$$

\vspace{1cm}
\begin{theorem}[Law of Total Probability] (denominator of Bayes' Theorem)\newline  

If the sample space $S$ can be split into disjoint events $A_1, A_2, ..., A_k$ that make up all possible outcomes in $S$, and if  $\mathbb{P}(A_i)>0$ for $i=1,\ldots, k$ and $\mathbb{P}(B)>0$, then


\[
%\left(
\begin{array}{ccl}
\mathbb{P}(B)&=& \mathbb{P}(B\ \textrm{and}\ A_1) + \mathbb{P}(B\ \textrm{and}\ A_2) + \ldots + \mathbb{P}(B\ \textrm{and}\ A_k)\\
		   &=& \mathbb{P}(B|A_1) \cdot \mathbb{P}(A_1) + \mathbb{P}(B|A_2) \cdot \mathbb{P}(A_2) + \ldots + \mathbb{P}(B|A_k) \cdot \mathbb{P}(A_k)
\end{array}
%\right)
\]

\vspace{0.5cm}

\end{theorem}


Special case of Law of Total Probability for sample space being split into $A$ and $A^c$: 

\[
%\left(
\begin{array}{ccl}
\mathbb{P}(B)&=&\mathbb{P}(B\ \textrm{and}\ A) + \mathbb{P}(B\ \textrm{and}\ A^C)\\
		   &=& \mathbb{P}(B|A) \cdot \mathbb{P}(A)+ \mathbb{P}(B | A^C)\cdot \mathbb{P}(A^C)
\end{array}
%\right)
\]


\newpage
%----------------------------------------------------------------------------


\begin{example} \textbf{Antibody test for COVID-19} 

According to the \href{https://www.fda.gov/medical-devices/coronavirus-disease-2019-covid-19-emergency-use-authorizations-medical-devices/eua-authorized-serology-test-performance}{FDA's EUA Authorized Serology Test Performance website}, %\newline
the Abbott AdviseDx SARS-CoV-2 IgG II (Alinity) antibody test for COVID-19 has sensitivity 98.1\% and PPV 98.4\% when the prevalence is 20\%. \newline

%Questions:\newline
Question: What is the specificity of the antibody test?

\vspace{0.5cm}
\textbf{What information were we given?}%\newline
%\vspace{0.5cm}

First, let's define our events of interest: 
\begin{itemize}
	\item $A$ = event one has antibodies for COVID-19
	\item $A^c$ = event one does not have antibodies
	\item $T^+$ = event one tests positive for antibodies
	\item $T^-$ = event one tests negative for antibodies
\end{itemize}
\vspace{0.5cm}
Translate given information into mathematical notation:
\begin{itemize}
	\item Sensitivity is 98.1\%:\newline
%	hide
	Sensitivity  = $\mathbb{P}(T^+|A) = 0.981$

\vspace{0.5cm}
	\item PPV is 98.4\%: \newline 
%	hide
	PPV = $\mathbb{P}(A|T^+) = 0.984$ 
	
\vspace{0.5cm}
	\item Prevalence is 20\%:\newline 
%	hide
	$\mathbb{P}(A) = 0.20$ 

\end{itemize}

\vspace{0.5cm}
\textbf{Solution:} %\newline
%	hide
Want specificity  = $\mathbb{P}(T^-|A^c) = 1 - \mathbb{P}(T^+|A^c)$

\vspace{0.5cm}
$$
\mathbb{P}(A | T^+) = 
\frac{\mathbb{P}(T^+|A) \cdot \mathbb{P}(A)}
{\mathbb{P}(T^+|A) \cdot \mathbb{P}(A) + \mathbb{P}(T^+|A^c) \cdot \mathbb{P}(A^c) }
$$

$$
0.984 = 
\frac{0.981 \cdot  0.20}
{0.981 \cdot  0.20 + \mathbb{P}(T^+|A^c) \cdot (1- 0.20) }
$$

$$
0.984 = 
\frac{0.1962}
{0.1962 + \mathbb{P}(T^+|A^c) \cdot 0.80 }
$$

$$
0.1962 = 0.984 \cdot(0.1962 + 0.80\mathbb{P}(T^+|A^c))
$$
$$
0.1962 = 0.1931 + 0.7872\cdot\mathbb{P}(T^+|A^c) 
$$
$$
\mathbb{P}(T^+|A^c)  = \frac{0.1962 - 0.984\cdot 0.1962}{  0.984\cdot 0.8} = 0.0039
$$
$$
\mathbb{P}(T^+|A^c)  = \frac{0.1962 - 0.1931}{  0.7872} = 0.00399
$$
specificity  = $\mathbb{P}(T^-|A^c) = 1 - \mathbb{P}(T^+|A^c)$ = 1 - 0.00399 = 0.99601

\end{example}
%----------------------------------------------------------------------------
}  % end large font
%----------------------------------------------------------------------------



\end{document}

